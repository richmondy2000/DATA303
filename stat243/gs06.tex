% Options for packages loaded elsewhere
\PassOptionsToPackage{unicode}{hyperref}
\PassOptionsToPackage{hyphens}{url}
%
\documentclass[
]{article}
\title{Stat 243 -- Gradescope 06}
\author{Richmond Yevudza}
\date{April 10, 2022}

\usepackage{amsmath,amssymb}
\usepackage{lmodern}
\usepackage{iftex}
\ifPDFTeX
  \usepackage[T1]{fontenc}
  \usepackage[utf8]{inputenc}
  \usepackage{textcomp} % provide euro and other symbols
\else % if luatex or xetex
  \usepackage{unicode-math}
  \defaultfontfeatures{Scale=MatchLowercase}
  \defaultfontfeatures[\rmfamily]{Ligatures=TeX,Scale=1}
\fi
% Use upquote if available, for straight quotes in verbatim environments
\IfFileExists{upquote.sty}{\usepackage{upquote}}{}
\IfFileExists{microtype.sty}{% use microtype if available
  \usepackage[]{microtype}
  \UseMicrotypeSet[protrusion]{basicmath} % disable protrusion for tt fonts
}{}
\makeatletter
\@ifundefined{KOMAClassName}{% if non-KOMA class
  \IfFileExists{parskip.sty}{%
    \usepackage{parskip}
  }{% else
    \setlength{\parindent}{0pt}
    \setlength{\parskip}{6pt plus 2pt minus 1pt}}
}{% if KOMA class
  \KOMAoptions{parskip=half}}
\makeatother
\usepackage{xcolor}
\IfFileExists{xurl.sty}{\usepackage{xurl}}{} % add URL line breaks if available
\IfFileExists{bookmark.sty}{\usepackage{bookmark}}{\usepackage{hyperref}}
\hypersetup{
  pdftitle={Stat 243 -- Gradescope 06},
  pdfauthor={Richmond Yevudza},
  hidelinks,
  pdfcreator={LaTeX via pandoc}}
\urlstyle{same} % disable monospaced font for URLs
\usepackage[margin=1in]{geometry}
\usepackage{graphicx}
\makeatletter
\def\maxwidth{\ifdim\Gin@nat@width>\linewidth\linewidth\else\Gin@nat@width\fi}
\def\maxheight{\ifdim\Gin@nat@height>\textheight\textheight\else\Gin@nat@height\fi}
\makeatother
% Scale images if necessary, so that they will not overflow the page
% margins by default, and it is still possible to overwrite the defaults
% using explicit options in \includegraphics[width, height, ...]{}
\setkeys{Gin}{width=\maxwidth,height=\maxheight,keepaspectratio}
% Set default figure placement to htbp
\makeatletter
\def\fps@figure{htbp}
\makeatother
\setlength{\emergencystretch}{3em} % prevent overfull lines
\providecommand{\tightlist}{%
  \setlength{\itemsep}{0pt}\setlength{\parskip}{0pt}}
\setcounter{secnumdepth}{-\maxdimen} % remove section numbering
\ifLuaTeX
  \usepackage{selnolig}  % disable illegal ligatures
\fi

\begin{document}
\maketitle

\hypertarget{section}{%
\subsubsection{9.10}\label{section}}

alpha = 0.05 with n - 2 = 30 - 2 = 28 = df

t* is 2.048

CI: (-0.7670, 0.0550)

\hypertarget{section-1}{%
\subsubsection{9.14}\label{section-1}}

\{r echo =TRUE\} varR \textless- -0.41 varN \textless- 18

(varR*sqrt(varN-2))/(sqrt(1-varR\^{}2))

\hypertarget{section-2}{%
\subsubsection{9.16}\label{section-2}}

a.) Points and steals have the highest positive correlation which is
0.453. The p-value is 0, and the positive correlation here means that an
increase in points or steals leads to an increase in the other value,
and a decrease in the number of points or steals will lead to a decrease
in the other value.

b.) The free throw shooting percentage and the number of rebounds mostly
negatively correlated, with a value of -0.384 and a p-value of 0. The
negative correlation here means that the higher the percentage of free
throws, the lower the number of rebounds, and the lower the percentage
of free throws, the higher the number of rebounds.

c.) Total list of p-values greater than 0.05:

Age/Points

FTPct/Age

Rebounds/Age

Steals/Age

Steals/FTPct

Steals/Rebounds

\hypertarget{section-3}{%
\subsubsection{9.18}\label{section-3}}

a.) No

b.) Correlation = 0.740; p-value = 0; reject the null hypothesis.

c.) BMGain = 1.11 + 0.127DayPct = 1.11 + 0.127(56) = 8.222 grams.

d.) 0.127, gives the strength of linear association between the two
variables.

e.) p-value of the slope is 0, so reject the null hypothesis.

f.) They are the same.

g.) \(R^2\) = 54.7\%

h.) \{r correlationSquared, include=TRUE\} 0.740\^{}2

\hypertarget{section-4}{%
\subsubsection{9.20}\label{section-4}}

a.) slope = 82.45, SE = 27.58

b.) t = 2.990

There is correlation between the normalized z-scores of grey matter
density in certain regions and the number of friends on Facebook.

p-value = 0.005

Therefore, reject the null hypothesis.

c.) n = 40

df = 40 - 2 = 38

t critical (95\% confidence): 2.024

95\% CI = (82.45 - (2.024 * 27.58), 82.45 + (2.024 * 27.58))

= (26.63, 138.27)

\hypertarget{section-5}{%
\subsubsection{9.23}\label{section-5}}

a.) \(R^2\)

b.) prevalence of the virus; explanatory: precipitation level

c.) 0.889

\hypertarget{section-6}{%
\subsubsection{9.28}\label{section-6}}

F: 1.75

p-value: 0.187

Therefore, model isn't effective because the P-value is greater than
0.05

\hypertarget{section-7}{%
\subsubsection{9.34}\label{section-7}}

n = 361

\(R^2\) = 0.02975085

\hypertarget{section-8}{%
\subsubsection{9.41}\label{section-8}}

F = 7.44

p-value = 0.011

Therefore, the model is effective because the P-value is less than 0.05

\hypertarget{section-9}{%
\subsubsection{9.43}\label{section-9}}

a.)GPA = 3.07

b.) n = 345

c.) \(R^2\) = 12.5\%

d.) F = 48.84; p-value = 0; the model is effective

\hypertarget{section-10}{%
\subsubsection{9.52}\label{section-10}}

a.) fitted regression: beds = 2.358512493 + 0.339337594 baths

substituting 3 into the equation:

beds (predicted in a house with 3 bathrooms) = 3.4

b.) t-stat: 1.978

p-value = 0.058 (which is greater than 0.05)

Therefore, fail to reject the null hypothesis.

c.) F-stat: 3.91

p-value: 0.058

Therefore, fail to reject the null hypothesis.

d.) \(R^2\) = 12.3\%

\hypertarget{section-11}{%
\subsubsection{9.62}\label{section-11}}

a.) 143.35, 172.42

b.) 101.46, 214.31

\hypertarget{section-12}{%
\subsubsection{9.63}\label{section-12}}

a.) p-value ≈ 0; effective predictor

b.) \$571,910

c.) (445.3, 698.6) in \$1000s

d.) (−133.2, 1277.0) in \$1000s

\hypertarget{section-13}{%
\subsubsection{9.66}\label{section-13}}

a.) 95\% CI = 5.3 ± (2.262 * 1.87)

= (1.08, 9.52)

b.) 95\% PI = 5.3 ± (2.262 * 6.24)

= (-8.81, 19.41)

\end{document}
