% Options for packages loaded elsewhere
\PassOptionsToPackage{unicode}{hyperref}
\PassOptionsToPackage{hyphens}{url}
%
\documentclass[
]{article}
\title{Stat 243 -- Homework 04}
\author{Richmond Yevudza}
\date{February 17, 2022}

\usepackage{amsmath,amssymb}
\usepackage{lmodern}
\usepackage{iftex}
\ifPDFTeX
  \usepackage[T1]{fontenc}
  \usepackage[utf8]{inputenc}
  \usepackage{textcomp} % provide euro and other symbols
\else % if luatex or xetex
  \usepackage{unicode-math}
  \defaultfontfeatures{Scale=MatchLowercase}
  \defaultfontfeatures[\rmfamily]{Ligatures=TeX,Scale=1}
\fi
% Use upquote if available, for straight quotes in verbatim environments
\IfFileExists{upquote.sty}{\usepackage{upquote}}{}
\IfFileExists{microtype.sty}{% use microtype if available
  \usepackage[]{microtype}
  \UseMicrotypeSet[protrusion]{basicmath} % disable protrusion for tt fonts
}{}
\makeatletter
\@ifundefined{KOMAClassName}{% if non-KOMA class
  \IfFileExists{parskip.sty}{%
    \usepackage{parskip}
  }{% else
    \setlength{\parindent}{0pt}
    \setlength{\parskip}{6pt plus 2pt minus 1pt}}
}{% if KOMA class
  \KOMAoptions{parskip=half}}
\makeatother
\usepackage{xcolor}
\IfFileExists{xurl.sty}{\usepackage{xurl}}{} % add URL line breaks if available
\IfFileExists{bookmark.sty}{\usepackage{bookmark}}{\usepackage{hyperref}}
\hypersetup{
  pdftitle={Stat 243 -- Homework 04},
  pdfauthor={Richmond Yevudza},
  hidelinks,
  pdfcreator={LaTeX via pandoc}}
\urlstyle{same} % disable monospaced font for URLs
\usepackage[margin=1in]{geometry}
\usepackage{color}
\usepackage{fancyvrb}
\newcommand{\VerbBar}{|}
\newcommand{\VERB}{\Verb[commandchars=\\\{\}]}
\DefineVerbatimEnvironment{Highlighting}{Verbatim}{commandchars=\\\{\}}
% Add ',fontsize=\small' for more characters per line
\usepackage{framed}
\definecolor{shadecolor}{RGB}{248,248,248}
\newenvironment{Shaded}{\begin{snugshade}}{\end{snugshade}}
\newcommand{\AlertTok}[1]{\textcolor[rgb]{0.94,0.16,0.16}{#1}}
\newcommand{\AnnotationTok}[1]{\textcolor[rgb]{0.56,0.35,0.01}{\textbf{\textit{#1}}}}
\newcommand{\AttributeTok}[1]{\textcolor[rgb]{0.77,0.63,0.00}{#1}}
\newcommand{\BaseNTok}[1]{\textcolor[rgb]{0.00,0.00,0.81}{#1}}
\newcommand{\BuiltInTok}[1]{#1}
\newcommand{\CharTok}[1]{\textcolor[rgb]{0.31,0.60,0.02}{#1}}
\newcommand{\CommentTok}[1]{\textcolor[rgb]{0.56,0.35,0.01}{\textit{#1}}}
\newcommand{\CommentVarTok}[1]{\textcolor[rgb]{0.56,0.35,0.01}{\textbf{\textit{#1}}}}
\newcommand{\ConstantTok}[1]{\textcolor[rgb]{0.00,0.00,0.00}{#1}}
\newcommand{\ControlFlowTok}[1]{\textcolor[rgb]{0.13,0.29,0.53}{\textbf{#1}}}
\newcommand{\DataTypeTok}[1]{\textcolor[rgb]{0.13,0.29,0.53}{#1}}
\newcommand{\DecValTok}[1]{\textcolor[rgb]{0.00,0.00,0.81}{#1}}
\newcommand{\DocumentationTok}[1]{\textcolor[rgb]{0.56,0.35,0.01}{\textbf{\textit{#1}}}}
\newcommand{\ErrorTok}[1]{\textcolor[rgb]{0.64,0.00,0.00}{\textbf{#1}}}
\newcommand{\ExtensionTok}[1]{#1}
\newcommand{\FloatTok}[1]{\textcolor[rgb]{0.00,0.00,0.81}{#1}}
\newcommand{\FunctionTok}[1]{\textcolor[rgb]{0.00,0.00,0.00}{#1}}
\newcommand{\ImportTok}[1]{#1}
\newcommand{\InformationTok}[1]{\textcolor[rgb]{0.56,0.35,0.01}{\textbf{\textit{#1}}}}
\newcommand{\KeywordTok}[1]{\textcolor[rgb]{0.13,0.29,0.53}{\textbf{#1}}}
\newcommand{\NormalTok}[1]{#1}
\newcommand{\OperatorTok}[1]{\textcolor[rgb]{0.81,0.36,0.00}{\textbf{#1}}}
\newcommand{\OtherTok}[1]{\textcolor[rgb]{0.56,0.35,0.01}{#1}}
\newcommand{\PreprocessorTok}[1]{\textcolor[rgb]{0.56,0.35,0.01}{\textit{#1}}}
\newcommand{\RegionMarkerTok}[1]{#1}
\newcommand{\SpecialCharTok}[1]{\textcolor[rgb]{0.00,0.00,0.00}{#1}}
\newcommand{\SpecialStringTok}[1]{\textcolor[rgb]{0.31,0.60,0.02}{#1}}
\newcommand{\StringTok}[1]{\textcolor[rgb]{0.31,0.60,0.02}{#1}}
\newcommand{\VariableTok}[1]{\textcolor[rgb]{0.00,0.00,0.00}{#1}}
\newcommand{\VerbatimStringTok}[1]{\textcolor[rgb]{0.31,0.60,0.02}{#1}}
\newcommand{\WarningTok}[1]{\textcolor[rgb]{0.56,0.35,0.01}{\textbf{\textit{#1}}}}
\usepackage{graphicx}
\makeatletter
\def\maxwidth{\ifdim\Gin@nat@width>\linewidth\linewidth\else\Gin@nat@width\fi}
\def\maxheight{\ifdim\Gin@nat@height>\textheight\textheight\else\Gin@nat@height\fi}
\makeatother
% Scale images if necessary, so that they will not overflow the page
% margins by default, and it is still possible to overwrite the defaults
% using explicit options in \includegraphics[width, height, ...]{}
\setkeys{Gin}{width=\maxwidth,height=\maxheight,keepaspectratio}
% Set default figure placement to htbp
\makeatletter
\def\fps@figure{htbp}
\makeatother
\setlength{\emergencystretch}{3em} % prevent overfull lines
\providecommand{\tightlist}{%
  \setlength{\itemsep}{0pt}\setlength{\parskip}{0pt}}
\setcounter{secnumdepth}{-\maxdimen} % remove section numbering
\ifLuaTeX
  \usepackage{selnolig}  % disable illegal ligatures
\fi

\begin{document}
\maketitle

\hypertarget{section}{%
\paragraph{3.14}\label{section}}

population mean = \textbf{300}

standard error = \(\frac{(310-290)}{4}=\) \textbf{5}

\hypertarget{a}{%
\paragraph{3.18a}\label{a}}

250 sample mean does not fall in the range of the sampling distribution
therefore \(\overline{x} = 250\) is \textbf{extremely unlikely to
occur}.

\hypertarget{b}{%
\paragraph{3.18b}\label{b}}

305 sample mean isn't uncommon in the sampling distribution therefore
\(\overline{x} = 305\) is \textbf{reasonably likely to occur from a
sample of this size}.

\hypertarget{c}{%
\paragraph{3.18c}\label{c}}

315 sample mean is uncommon in the sampling distribution but the same
sample mean takes place quite frequently in this sampling distribution
therefore \(\overline{x} = 315\) is \textbf{unusual but might occur
occasionally}.

\hypertarget{a-1}{%
\paragraph{3.23a}\label{a-1}}

30 dollars shows the average co-payment for month's supply of ointment
for regular users. This is also a parameter and the notation for
population mean is

27.90 dollars is the sample mean co-payment of regular users of sample
size 75. This is a statistic and the notation for sample statistic is
\(\mu\). \(\overline{x}\).

\hypertarget{b-1}{%
\paragraph{3.23b}\label{b-1}}

The shape of the dot plot would be \textbf{bell shaped} because there
will be one point of sample mean 27.90. The plot will also be centered
at the average of \(\mu\)=30.

\hypertarget{c-1}{%
\paragraph{3.23c}\label{c-1}}

There will be 1000 dots in the dot plot because there are 1000 samples.
Each dot shows the sample mean of 75 co-payments.

\hypertarget{a-2}{%
\paragraph{3.24a}\label{a-2}}

It is given that the average household size in US is 2.61. The sampling
distribution which is not centered at the average 2.61 is a biased
distribution.

For unbiased distributions, we are looking at distributions A and D
since the sampling distribution for these two are centered at 2.61 (the
average).

The distribution B has more dots at the right of the center 2.61, which
produces a higher average. Therefore, distribution B is a biased
distribution. The distribution C has more dots at the left of the center
2.61 which produces a higher average. Therefore, distribution C is also
a biased distribution.

\hypertarget{b-2}{%
\paragraph{3.24b}\label{b-2}}

Variability of distribution decreases with increase in sample size.
Given that one distribution has the sample size n=100 and the other has
the sample size n=500, looking at the sets of 1000 sample means, the
distribution A has more variability and distribution D has less
variability. Therefore, according to sample sizes n=100 is of the
distribution A and n=500 is of distribution D.

\hypertarget{section-1}{%
\paragraph{3.42}\label{section-1}}

Interval estimate = \(5 \pm 8 = (5-8, 5+8) = (-3,13)\)

The interval estimate of the difference of population mean
\(\mu_1-\mu_2\) is \textbf{-3 to 13}.

\hypertarget{a-3}{%
\paragraph{3.51a}\label{a-3}}

30\% is a \textbf{statistic}.

The correct notation of the sample proportion is \(\hat{p}\).

\hypertarget{b-3}{%
\paragraph{3.51b}\label{b-3}}

The estimated proportion p is 0.30 by using the sample information which
in this case was estimated with the information, the young people who
were arrested by the age of 23 in US.

\hypertarget{c-2}{%
\paragraph{3.51c}\label{c-2}}

Interval estimate =
\(0.30 \pm 0.01 = (0.30-0.01, 0.30+0.01) = (0.29,0.31)\)

The interval estimate of population proportion p is \textbf{0.29 to
0.31}.

\hypertarget{d}{%
\paragraph{3.51d}\label{d}}

The actual proportion can't be less than 25\% because the range of
reasonable values for the population proportion is 0.29 to 0.31.

\hypertarget{a-4}{%
\paragraph{3.52a}\label{a-4}}

Population: All people in the US whose ages are 18 or above 18.

Sample: The random sample of 147291 adults who were in touch with and
inquire whether they got health insurance from an employer or not.

Relevant statistic: \(\hat{p}=0.45\)

\hypertarget{b-4}{%
\paragraph{3.52b}\label{b-4}}

Interval estimate =
\(0.45 \pm 0.01 = (0.45-0.01, 0.45+0.01) = (0.44,0.46)\)

\hypertarget{section-2}{%
\paragraph{3.53}\label{section-2}}

It is 95\% confident that proportion of all adults who believe the
necessity of a car is between 0.83 and 0.89.

\hypertarget{a-5}{%
\paragraph{3.58a}\label{a-5}}

This is a matched pairs experiment because in the experiment of urinary
BPA concentration all the participants were used in both treatments.

In the experiment it is seen that there is a large variability in BPA
concentration so the matched pair experiment is used to reduce
variability of the data.

\hypertarget{b-5}{%
\paragraph{3.58b}\label{b-5}}

\(\mu_1-\mu_2\)

\hypertarget{c-3}{%
\paragraph{3.58c}\label{c-3}}

It is 95\% confident that the difference in the mean urinary BPA
concentration after taking canned soup and fresh soup for five days will
be between 19.6\(\mu\)g/L and 25.5\(\mu\)g/L.

\hypertarget{d-1}{%
\paragraph{3.58d}\label{d-1}}

The sample mean difference will be close to population mean difference
if a large sample is taken. This concludes that the confidence interval
will be narrower if the study includes 500 participants instead of 75.

\hypertarget{a-6}{%
\paragraph{3.60a}\label{a-6}}

The relevant parameter is the \textbf{population mean} denoted by
\(\mu\). This is the mean effect on weight over two years after one
month overeating.

\hypertarget{b-6}{%
\paragraph{3.60b}\label{b-6}}

To find the actual exact value of the population mean, we have to
collect all population who overeat for one month. Then after that, we
find out the result after two and a half years.

\hypertarget{c-4}{%
\paragraph{3.60c}\label{c-4}}

Confidence interval =
\(6.8 \pm 2*1.2 = (6.8-2 * 1.2, 6.8+2*1.2) = (4.4,9.2)\)

Therefore, it is 95\% confident that the true population mean effect on
weight put on over two years after one month of overeating lies between
\textbf{4.4 and 9.2 pounds}.

\hypertarget{d-2}{%
\paragraph{3.60d}\label{d-2}}

Margin of error = \(2*1.2 = 2.4\)

Therefore, it would be expected that estimate of the mean 6.8 pounds
lies within 2.4 pounds of true mean.

\hypertarget{a-7}{%
\paragraph{3.64a}\label{a-7}}

The given interpretation of the 95\% confidence interval is not correct
because 95\% confidence interval is for the population mean pulse rate
not for the sample mean pulse rate of the students.

\hypertarget{b-7}{%
\paragraph{3.64b}\label{b-7}}

The given interpretation of the 95\% confidence interval is not correct
because 95\% confidence interval is for the population mean pulse rate
not for the sample mean pulse rate of the students

\hypertarget{c-5}{%
\paragraph{3.64c}\label{c-5}}

The given interpretation of the 95\% confidence interval is not in doubt
because it is not sure whether or not the interval considers the
population mean.

\hypertarget{d-3}{%
\paragraph{3.64d}\label{d-3}}

The given interpretation of the 95\% confidence interval is not correct
because 95\% confidence interval for the mean pulse rate is not from the
US college students.

\hypertarget{e}{%
\paragraph{3.64e}\label{e}}

The given interpretation of the 95\% confidence interval is not correct
because 95\% confidence interval for the mean pulse rate is not from the
US college students.

\hypertarget{f}{%
\paragraph{3.64f}\label{f}}

The given interpretation of the 95\% confidence interval is not correct
because the population mean is a particular permanent value.

\hypertarget{g}{%
\paragraph{3.64g}\label{g}}

The given interpretation of the 95\% confidence interval is not correct
because 95\% confidence interval is for the population mean pulse rate
not for the sample mean pulse rate of the students.

\hypertarget{a-8}{%
\paragraph{3.66a}\label{a-8}}

The sample 79, 79, 97, 85, 88 is the possible bootstrap sample from the
original sample since all the values are there in the original sample
and also the sample size of the given sample and the original sample is
same.

\hypertarget{b-8}{%
\paragraph{3.66b}\label{b-8}}

The sample 72, 79, 85, 88, 97 is the possible bootstrap sample from the
original sample since all the values are there in the original sample
and also the sample size of the given sample and the original sample is
same.

\hypertarget{c-6}{%
\paragraph{3.66c}\label{c-6}}

The sample 85, 88, 97, 72 is not the possible bootstrap sample from the
original sample since the sample size of the bootstrap sample and the
original sample should be same. Here, the sample size of the given
sample is 4 and the sample size of the original sample is 5.

\hypertarget{d-4}{%
\paragraph{3.66d}\label{d-4}}

The sample 88, 97, 81, 78, 85 is not the possible bootstrap sample from
the original sample since the original sample does not have the value
78.

\hypertarget{e-1}{%
\paragraph{3.66e}\label{e-1}}

The sample 97, 85, 79, 85, 97 is the possible bootstrap sample from the
original sample since all the values are there in the original sample
and also the sample size of the given sample and the original sample is
same.

\hypertarget{f-1}{%
\paragraph{3.66f}\label{f-1}}

The sample 72, 72, 79, 72, 79 is the possible bootstrap sample from the
original sample since all the values are there in the original sample
and also the sample size of the given sample and the original sample is
same.

\hypertarget{a-9}{%
\paragraph{3.75a}\label{a-9}}

\(\hat{p} = \frac{26}{174} = 0.149\)

Therefore the best estimate of the proportion of all snails of this type
to live after being eaten by a bird is \textbf{0.149}.

\hypertarget{b-9}{%
\paragraph{3.75b}\label{b-9}}

SE = 0.028

\hypertarget{c-7}{%
\paragraph{3.75c}\label{c-7}}

Confidence interval = \((\hat{p_1} - \hat{p_a}) \pm 2 *SE\)

Confidence interval =
\(0.149\pm 2*0.028 =(0.149-0.056,0.149+0.056) = (0.093,0.205)\)

Therefore, it is 95\% confident that the true proportion of all snails
that will live after being eaten by a bird lies between \textbf{0.093
and 0.205}.

\hypertarget{a-10}{%
\paragraph{3.76a}\label{a-10}}

s = \(\sqrt{\frac{\sum_{i=1}ˆn (x_i-\overline{x})^2}{n-1}}\)

s = \(\sqrt{\frac{1498}{7}} = \sqrt{214} = 14.63\)

Therefore, the standard deviation of the sample is \textbf{14.63}.

\hypertarget{b-10}{%
\paragraph{3.76b}\label{b-10}}

To create the bootstrap shows the eight slips by eight values and mingle
those slips. Take one slip and write the number and put that slip back.
Do this step for eight times. The eight numbers is the bootstrap
samples. The bootstrap statistic is the mean of the eight numbers.

\hypertarget{c-8}{%
\paragraph{3.76c}\label{c-8}}

We expect the bootstrap distribution to be \textbf{bell shaped} and it's
center will be at \textbf{34}.

\hypertarget{d-5}{%
\paragraph{3.76d}\label{d-5}}

The population parameter of interest is population mean \(\mu\). That is
the average number of ants that will climb on a piece of a peanut butter
sandwich left on the ground near an ant hill. The best point estimate of
the population mean is \(\overline{x}\) = 34

\hypertarget{e-2}{%
\paragraph{3.76e}\label{e-2}}

Confidence interval = \((\hat{p_1} - \hat{p_a}) \pm 2 *SE\)

Confidence interval = \(34\pm 2*4.85 =(34-4.85,34+4.85) = (24.3,43.7)\)

\hypertarget{section-3}{%
\paragraph{3.78**}\label{section-3}}

\(\overline{x} = \frac{658+456+830+385}{5} = \frac{3025}{5} = 605\)

Therefore, the mean of the sample is\textbf{605}.

Confidence interval = \((\hat{p_1} - \hat{p_a}) \pm 2 *SE\)

Confidence interval =
\(605\pm 2*70.597 =(605-141.194,605+141.194) = (463.81,746.19)\)

Therefore, it is 95\% confident that the average monthly sales in the
United States lies between \textbf{463.81 and 746.19}.

\hypertarget{a-11}{%
\paragraph{3.81a}\label{a-11}}

The correct option is 0.015. This is because the figure of the bootstrap
difference in sample proportions of teen and adult cell phone users who
text shows that distribution is centered lying on the value of the
difference in population proportion. This means that the estimate of the
difference in the population proportion is \(p_1 - p_a = 0.15\).

The figure of the bootstrap difference in sample proportions of teen and
adult cell phone users who text shows that the central 95\% of
difference in sample proportions show in a range from 0.12 to 0.18. This
range should distance about two standard errors below the mean and two
standard errors above the mean.

Therefore, the estimate of the standard error is around
\(\frac{(0.18-0.12)}{4}= 0.015\)

\hypertarget{b-11}{%
\paragraph{3.81b}\label{b-11}}

Confidence interval = \((\hat{p_1} - \hat{p_a}) \pm 2 *SE\)

Confidence interval =
\(0.15\pm 2*0.015 =(0.15-2 * 0.015,0.15+2 * 0.015) =(0.15-0.03,0.15 + 0.03) = (0.12,0.18)\)

Therefore, it is 95\% confident that the difference in proportions of
teen and adult cell phone users who text will lie between 0.12 and 0.18.

\hypertarget{a-12}{%
\paragraph{3.84a}\label{a-12}}

\textbf{Mean (Difference)}

\begin{Shaded}
\begin{Highlighting}[]
\FunctionTok{mean}\NormalTok{(}\SpecialCharTok{\textasciitilde{}}\NormalTok{Distance, }\AttributeTok{data=}\NormalTok{CommuteAtlanta)}
\end{Highlighting}
\end{Shaded}

\begin{verbatim}
## [1] 18.156
\end{verbatim}

\textbf{Standard Deviation (Difference)}

\begin{Shaded}
\begin{Highlighting}[]
\FunctionTok{sd}\NormalTok{(}\SpecialCharTok{\textasciitilde{}}\NormalTok{Distance, }\AttributeTok{data=}\NormalTok{CommuteAtlanta)}
\end{Highlighting}
\end{Shaded}

\begin{verbatim}
## [1] 13.79828
\end{verbatim}

\hypertarget{b-12}{%
\paragraph{3.84b}\label{b-12}}

The shape is bell shaped that is centered at 18.156.

\hypertarget{c-9}{%
\paragraph{3.84c}\label{c-9}}

Looking at the upper right part of the graph, the standard error for
mean commujte distance for 1000 bootstrap sample is \textbf{0.606}.

\hypertarget{d-6}{%
\paragraph{3.84d}\label{d-6}}

Confidence interval = \((\hat{p_1} - \hat{p_a}) \pm 2 *SE\)

Confidence interval =
\(18.6\pm 2*0.60 =(18.6-1.212,18.6+1.212) = (16.948,19.372)\)

Therefore, it is 95\% confident that the mean of the commute distances
in Commute Atlanta lies between \textbf{16.948 and 19.372 miles}.

\hypertarget{a-13}{%
\paragraph{3.89a}\label{a-13}}

The maximum confidence level would be 100\%. This means that, keeping
95\% of the middle value, the remaining percentage left is (100-95)\% =
5\%. This 5\% should be divided equally into two tails. The value that
must be chopped off from each tail for 95\% confidence level is
2.5\%.The bootstrap distribution contains 1,000 bootstrap samples. That
is, 2.5\% of 1.000 = 25 of the lowest and highest values should be
chopped off to create 95\% confidence interval

\hypertarget{b-13}{%
\paragraph{3.89b}\label{b-13}}

The maximum confidence level would be 100\%. This means that, keeping
90\% of the middle value, the remaining percentage left is (100-90)\%
=10\%. This 10\% should be divided equally into two tails. The value
that must be chopped off from each tail for 90\% confidence level is
5\%. The bootstrap distribution contains 1,000 bootstrap samples. That
is, 5\% of 1.000 = 50 of the lowest and highest values should be chopped
off to create 90\% confidence interval.

\hypertarget{c-10}{%
\paragraph{3.89c}\label{c-10}}

The maximum confidence level would be 100\%. This means that, keeping
98\% of the middle value, the remaining percentage left is (100-98)\%
=2\%. This 2\% should be divided equally into two tails. The value that
must be chopped off from each tail for 98\% confidence level is 1\%. The
bootstrap distribution contains 1,000 bootstrap samples. That is, 1\% of
1,000 = 10 of the lowest and highest values should be chopped off to
create 98\% confidence interval.

\hypertarget{d-7}{%
\paragraph{3.89d}\label{d-7}}

The maximum confidence level would be 100\%. This means that, keeping
99\% of the middle value, the remaining percentage left is (100-99)\%
=1\%. This 1\% should be divided equally into two tails. The value that
must be chopped off from each tail for 99\% confidence level is 0.5\%.
The bootstrap distribution contains 1,000 bootstrap samples. That is,
0.5\% of 1.000 = 5 of the lowest and highest values should be chopped
off to create 99\% confidence interval.

\hypertarget{section-4}{%
\paragraph{3.90}\label{section-4}}

\textbf{Option A: ``66 to 74'' is the most likely result after the
change}. This is because the length of the confidence interval increases
with the increment of the confidence level.

Therefore, 99\% confidence interval should be 66 to 74.

\hypertarget{section-5}{%
\paragraph{3.91}\label{section-5}}

\textbf{Option C: ``67.5 to 72.5'' is the most likely result after the
change}. This is because the length of the confidence interval decreases
with the decrease of the confidence level.

Therefore, 90\% confidence interval should be 67.5 to 72.5.

\hypertarget{section-6}{%
\paragraph{3.92}\label{section-6}}

\textbf{Option C: ``67.5 to 72.5'' is the most likely result after the
change}. This is because the standard error decreases with the increase
of the sample size. Thus, the bootstrap distribution will be less
widened, so the confidence interval will be of small width.

Therefore, using the sample size n = 45, the 95\% confidence interval
should be 67.5 to 72.5

\hypertarget{section-7}{%
\paragraph{3.93}\label{section-7}}

\textbf{Option A: ``66 to 74'' is the most likely result after the
change}. This is because the standard error increases with the decrease
of the sample size. Thus, the bootstrap distribution will be more
widened, so the confidence interval will be of larger width.

Therefore, using the sample size n= 16, the 95\% confidence interval
should be 66 to 74.

\hypertarget{section-8}{%
\paragraph{3.94}\label{section-8}}

\textbf{Option B: ``67 to 73'' is the most likely result}. This is
because the width of the confidence interval will not change by adding
or subtracting bootstrap samples

\hypertarget{section-9}{%
\paragraph{3.95}\label{section-9}}

\textbf{Option B: ``67 to 73'' is most likely result}. This is because
the width of the confidence interval will not change by adding or
subtracting bootstrap samples.

\hypertarget{a-14}{%
\paragraph{3.100a}\label{a-14}}

From the Figure of the bootstrap distribution of sample means of IQ
scores it is clear that the distribution has its center at 100.

Therefore, the mean of the original sample of IQ scores is
\(\overline{x}=100\).

\hypertarget{b-14}{%
\paragraph{3.100b}\label{b-14}}

The maximum confidence level would be 100\%. This means that keeping
99\% of the middle value, the remaining percentage left is (100-99)\% =
1\%. This 1\% should be divided equally into two tails. The values to be
removed from each tail for 99\% confidence level are 0.5\%. The
bootstrap distribution contains 1,000 bootstrap samples. That is, 0.5\%
of 1,000 = 5 of the lowest and highest values should be removed to
create 99\% confidence interval.

Therefore, the 99\% confidence interval is \textbf{(88,112)}.

\hypertarget{section-10}{%
\paragraph{3.102}\label{section-10}}

Here for the 90\% confidence interval,z valueis 1.645 Margin of error =
\(\sqrt(0.753*(1-0.753)/1000)= 1.645 * 0.0136= 0.022\)

New Interval = (0.753 -0.022, 0.753+ 0.022)= \textbf{(0.731,0.775)}.

\hypertarget{section-11}{%
\paragraph{3.105}\label{section-11}}

Confidence interval = \((\hat{p_1} - \hat{p_a}) \pm 2 *SE\)

Confidence interval =
\(3.860\pm 2*0.189 =(3.860-2 * 0.189,3.860+2 * 0.189) =(3.860-0.378,3.860 + 0.378) = (3.48,4.23)\)

Therefore, the 95\% confidence interval for the average tip left at the
restaurant is \textbf{(3.48, 4.23)}.

\hypertarget{a-15}{%
\paragraph{6.4a}\label{a-15}}

SE =
\(\sqrt{\frac{p(1-p)}{n}}=\sqrt{\frac{0.27(1-0.27)}{30}}= \sqrt{\frac{0.1971}{30}} = 0.081\)

Therefore the standard error of the distribution of sample proportion is
\textbf{0.081}.

\hypertarget{b-15}{%
\paragraph{6.4b}\label{b-15}}

\(np = 30(0.27) = 8.1 [<10]\)

\(n(1-p) = 30(1-0.27) = 21.9 [>10]\)

Since the value of np is less than 10 and n(1-p) is greater than 10, the
distribution sample proportion isn't reasonably normal.

\hypertarget{a-16}{%
\paragraph{6.10a}\label{a-16}}

SE =
\(\sqrt{\frac{p(1-p)}{n}}=\sqrt{\frac{0.69(1-0.69)}{100}}= \sqrt{\frac{0.2139}{100}} = 0.046\)

Therefore the standard error of the distribution of sample proportion is
\textbf{0.046}.

\hypertarget{b-16}{%
\paragraph{6.10b}\label{b-16}}

SE =
\(\sqrt{\frac{p(1-p)}{n}}=\sqrt{\frac{0.69(1-0.69)}{1000}}= \sqrt{\frac{0.2139}{1000}} = 0.015\)

Therefore the standard error of the distribution of sample proportion is
\textbf{0.015}.

\hypertarget{c-11}{%
\paragraph{6.10c}\label{c-11}}

SE =
\(\sqrt{\frac{p(1-p)}{n}}=\sqrt{\frac{0.75(1-0.75)}{100}}= \sqrt{\frac{0.1875}{100}} = 0.043\)

Therefore the standard error of the distribution of sample proportion is
\textbf{0.043}.

\hypertarget{d-8}{%
\paragraph{6.10d}\label{d-8}}

SE =
\(\sqrt{\frac{p(1-p)}{n}}=\sqrt{\frac{0.75(1-0.75)}{1000}}= \sqrt{\frac{0.1875}{1000}} = 0.014\)

Therefore the standard error of the distribution of sample proportion is
\textbf{0.014}.

\hypertarget{section-12}{%
\paragraph{6.12}\label{section-12}}

\textbf{p = 0.8}

SE =
\(\sqrt{\frac{p(1-p)}{n}}=\sqrt{\frac{0.8(1-0.8)}{100}}= \sqrt{\frac{0.16}{100}} = 0.040\)

Therefore the standard error of the distribution of sample proportion is
\textbf{0.040}.

\textbf{p = 0.5}

SE =
\(\sqrt{\frac{p(1-p)}{n}}=\sqrt{\frac{0.5(1-0.5)}{100}}= \sqrt{\frac{0.25}{100}} = 0.050\)

Therefore the standard error of the distribution of sample proportion is
\textbf{0.050}.

\textbf{p = 0.3}

SE =
\(\sqrt{\frac{p(1-p)}{n}}=\sqrt{\frac{0.3(1-0.3)}{100}}= \sqrt{\frac{0.21}{100}} = 0.046\)

Therefore the standard error of the distribution of sample proportion is
\textbf{0.046}.

\textbf{p = 0.1}

SE =
\(\sqrt{\frac{p(1-p)}{n}}=\sqrt{\frac{0.1(1-0.1)}{100}}= \sqrt{\frac{0.09}{100}} = 0.030\)

Therefore the standard error of the distribution of sample proportion is
\textbf{0.030}.

The largest standard error is \textbf{0.050 for p = 0.5} and the
smallest standard error is \textbf{0.030 for p = 0.1}.

\hypertarget{a-17}{%
\paragraph{6.16a}\label{a-17}}

\hypertarget{b-17}{%
\paragraph{6.16b}\label{b-17}}

SE =
\(\sqrt{\frac{p(1-p)}{n}}=\sqrt{\frac{0.4(1-0.4)}{100}}= \sqrt{\frac{0.24}{100}} = 0.049\)

Therefore the standard error of the distribution of sample proportion is
\textbf{0.049}. This shows us that the results obtained from part (a) is
similar to the results obtained in part (b).

\hypertarget{a-18}{%
\paragraph{6.18a}\label{a-18}}

\hypertarget{b-18}{%
\paragraph{6.18b}\label{b-18}}

\(np = 10(0.2) = 2 [<10]\)

\(n(1-p) = 10(1-0.2) = 8 [>10]\)

Since the value of np is less than 10 and n(1-p) is greater than 10, the
distribution sample proportion doesn't follow normal distribution.

SE =
\(\sqrt{\frac{p(1-p)}{n}}=\sqrt{\frac{0.2(1-0.2)}{10}}= \sqrt{\frac{0.16}{10}} = 0.126\)

Therefore the standard error of the distribution of sample proportion is
\textbf{0.126}. This shows us that the results obtained from part (a) is
slightly closer to the results obtained in part (b).

\end{document}
